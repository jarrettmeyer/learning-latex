\documentclass{article}


\usepackage{geometry}
\usepackage{eulervm}
\usepackage{amsmath}
\usepackage{listings}
\usepackage{xcolor}
\usepackage[usefilenames]{plex-otf}
\usepackage{biblatex}


% Add bibliography file.
\addbibresource{references.bib}

\title{Adding Code to a Document}
\author{Jarrett Meyer}
\date{August 3, 2020}

\geometry{margin=1in}
\setlength{\parskip}{1em}

\definecolor{r-background-color}{HTML}{F8F8F8}
\definecolor{r-comment-color}{HTML}{40772D}
\definecolor{r-keyword-color}{HTML}{007AD4}
\definecolor{r-string-color}{HTML}{FF8000}

\lstdefinestyle{custom-r}{
	language={R},
	frame={tblr},
	framesep={4pt},
	framerule={0pt},
	deletekeywords={_},
	morekeywords={in},
	basicstyle=\ttfamily,
	backgroundcolor=\color{r-background-color},
	commentstyle=\color{r-comment-color},
	keywordstyle=\color{r-keyword-color}\bfseries,
	stringstyle=\color{r-string-color},
	showstringspaces=false,
}

% Begin the primary document.
\begin{document}

\maketitle

The following function originally appears as Example 1.3.3 in Introduction to Mathematical Statistics \cite{Hogg2019}. I have cleaned it up from its original version. The original R code is, in my opinion, grossly written, and it should never have been presented to the reader, who may be reading R code for the very first time. My code has added the following features that were missing from the original version.

\begin{itemize}
	\item error handling for the input variable
	\item better error handling
	\item more descriptive variable names
	\item proper use of the traditional R assignment operator, \texttt{<-} vs \texttt{=}
	\item use of the \texttt{seq} function vs the \texttt{:} operator
\end{itemize}

% Show how to include 
\lstinputlisting[style={custom-r}, caption={Computing probabilities of like birthdays}, label={lst:bday}]{../r-src/prob_bday.R}

You may call this function from the R prompt in the following manner.

\begin{lstlisting}[style={custom-r}]
> source("prob_bday.R")
> prob_bday(10)
[1] 0.1169482
\end{lstlisting}

The proof for the above R function is follows.

Let $P(S_n)$ be the probability that in a room of $n$ people, there are at least 2 people with the same birthday. Let $P(D_n)$ be the complement of $P(S_n)$ such that $P(S_n) = 1 - P(D_n)$.

In a group of $n$ people, there are $P^{365}_n$ combinations of where everyone has a different birthday. Assuming that the probability of any given birthday is $\frac{1}{365}$ and assuming that birthdays are independent. Therefore, for $n \geq 2$,

\begin{align*}
	P(S_n) &= 1 - \frac{P^{365}_n}{365^n} \\
	P(S_n) &= 1 - \frac{\frac{365!}{(365 - n)!}}{365^n}
\end{align*}

We can see the following pattern emerge.

\begin{align*}
	P(S_1) & = 1 - \frac{364}{365} \\
	P(S_2) & = 1 - \frac{(364)(363)}{(365)(365)} \\
	P(S_3) & = 1 - \frac{(364)(363)(362)}{(365)(365)(365)} \\
	P(S_4) & = 1 - \frac{(364)(363)(362)(361)}{(365)(365)(365)(365)} \\
	\dots  &
\end{align*}

This is the result that given in Listing \ref{lst:bday}, shown above. 

\clearpage
\printbibliography

\end{document}
